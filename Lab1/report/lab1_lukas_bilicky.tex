\documentclass[a4paper,12pt]{article}
\usepackage[a4paper,top=1.5cm,bottom=1.5cm,left=1.5cm,right=1.5cm]{geometry}
\usepackage{graphicx}
\usepackage{wrapfig}

\title{Lab 1 Report}
\author{Lukas Bilicky}
\date{\today}

\begin{document}

\maketitle

\section{Methodology Used}
For the methodology, I used the structure described in the lab assignment. The lab starts with an exploratory data analysis which clarifies the data, allows us to remove uncorrelated or redundant attributes, and reveals missing data points. The next step was to cluster the patients using a clustering algorithm. In the last step two models were fitted: a Kaplan-Meier and a Cox model.

\subsection{Exploratory Data Analyisis (EDA)}

The EDA consisted of multiple analysis steps including plotting the value distrubution per attribute and the attribute's influence on the target attribute (survival\_status). The distribution of the values of time-based attributes revealed that the value 1 000 000 was used if the event had not occured. This would wrongly influence the data during the scaling so I extracted the values into its own column called "\{ColumnName\}\_never\_occured". The final step was the generation of a PHIK correlation matrix which is a more robust version of a classical correlation matrix which additionally contains also the categorical attributes. This revealed which attributes are not correlated with the target attribute and were removed. The PHIK matrix also revealed which attributes are redundant because they were almost 100\% correlated between each other.

\subsection{Clustering}

For clustering the KMeans algorithm was used because I found it well interpretable. The clusters were created based on the cleaned and preprocessed data: filled missing values, scaled numerical features, one-hot encoded categorical features. Furthermore, the Principal component analysis (PCA) dimensionality reduction was applied on the data before creating the clusters to make them more easily visualizable, remove further redundant attributes, and to help the clustering algorithm. The number of clusters was based on an "elbow" method where the inertia of the clusters for a $k$ is followed and when it starts to decrease more slowly (reaches the elbow), any larger $k$ would lead to diminishing returns.

\section{Discussion of Results and Challenges Encountered}

\begin{figure}[h]
  \centering
  \begin{minipage}{0.48\textwidth}
    \centering
    \includegraphics[width=\linewidth]{../results/km_survival_curves.png}
    \caption{Kaplan-Meier Survival Curves}
  \end{minipage}
  \begin{minipage}{0.48\textwidth}
    \centering
    \includegraphics[width=\linewidth]{../results/cox_predicted_survival.png}
    \caption{Cox Predicted Survival}
  \end{minipage}\hfill
\end{figure}

For the discussion of the results I decided to pick three clusters according to the projected survival time. The clusters number three, five and seven, being the ones with the respectably best, intermediate, and the worst survival time predictions.

The most notable characteristics of the best cluster, cluster number three, were the following:

\begin{itemize}
  \item Risk Group: every patient in this group was classified in the low-risk category.
  \item Acute Graft versus Host Disease (aGvHD) grade III-IV, extensive chronic GvHD: the patients had zero occurences of either of these two complications.
  \item Relapse: no patients had undergone a relapse.
  \item Patient age: the patients were young with a mean age of five with the oldest patient being 10.
  \item Platelet (PLT) Recovery: this group interestingly took the most time on average to recover their platelet's count but all patients did recover them which cannot be said about other groups except group one.
\end{itemize}

The intermediate group (number five) has a projected survival rate of 50\% after 500 days. These are some of the attributes of this group:

\begin{itemize}
  \item Patient Age: the patients were little older than in group three, having 70\% of recipients over age of ten.
  \item Relapse: this is the group with the highest occurence of relapse with 25\% rate.
  \item aGvHD and extcGvHD: these complications occured to 15-20\% of the patients.
\end{itemize}

The group with the lowest survival chances was the group seven with the following characteristics:

\begin{itemize}
  \item
\end{itemize}

\begin{wrapfigure}{r}{0.5\linewidth}
  \centering
  \includegraphics[width=0.9\linewidth]{../results/kmeans_pca_clusters.png}
  \caption{KMeans Clusters Visualized with PCA}
\end{wrapfigure}

\section{Conclusion}
Summary, conclusion, lessons learned?

\end{document}
